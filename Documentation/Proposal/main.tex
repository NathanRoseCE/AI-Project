\documentclass[11pt]{article}

\usepackage[backend=biber,style=ieee]{biblatex}
\usepackage{hyperref}
\hypersetup{
    colorlinks=true,
}
\addbibresource{resources.bib}

\title{AI Poker}
\author{Nathan, JP}
\date{\today}

\begin{document}
\maketitle
\section{Problem Statement}
200 words max
\begin{enumerate}
\item Clearly describes the problem being addressed
\item References any external resources that support this problem (e.g. data sets, simulation/game environment,
  etc.)
\end{enumerate}

The goal of the project is to train an AI to be very good at poker(Texas Holdem \cite{TexasHoldEm}),
in order to do this a poker game will be created. This game will allow human vs AI(for testing) and AI vs AI(
for training). In order to simplify the training, the game will only feature 5 players.

\section{Proposed Approach}
300 words max
\begin{enumerate}
\item Identifies the type of AI/ML approach to be used to address the problem, e.g.
  \begin{enumerate}
  \item Search
  \item Classification
  \item Prediction/regression
  \item Reinforcement Learning
  \end{enumerate}
\item Describe how your problem relates to the proposed approach (i.e. how do you map the problem to the
  approach?)
\item Identify any planned libraries, resources, etc. you intend to utilize to facilitate your approach
\end{enumerate}
In order to train the model, the system will be forced to play against itself in order to improve its skills,
this unsupervised with reinforcement learning approach is used as the system does not have labeled correct
plays, due to hidden information. One possible method of training could be NEAT\cite{NEAT} where neurons are
added in between generations. The AI will be rewarded based on gaining the most money overe a set number of rounds.

To train the AI will be given information about its hand, the cards currently available, the pot ammount, and
each player's bet ammount. It will be evaluated based off the money the AI has after 10 rounds.

\section{Team Structure}
\begin{enumerate}
\item Identifies team members
\item Identifies any relevant background of team member(s) to the problem, if applicable
\item Identifies roles and responsibilities of team members
\end{enumerate}

We are using AGILE SCRUM to assign tasks to maintain flexibility over time. Issues will be tracked on github and
we will make sure we are logging who closed issues for end evalution.
\subsection{Team Members with backgrounds}
\begin{enumerate}
\item Nathan 
  \begin{enumerate}
  \item Helped architect and create several projects in Industry
  \item Experience leading software teams and setting up project infrastructure
  \item Obviously AGILE SCRUM experience.
  \item has watched youtube videos on AI, but never done one.
  \end{enumerate}
\item JP
  \begin{enumerate}
  \item Took a genetic algorithms class abroad during my undergrad
  \item Masters Assistantship Contract project with the FAA focused on AI/ML (currently in progress)
  \item Passion and understanding of Texas Hold'em poker
  \item Experience with with system design, implementation and testing complete process/lifecycle
  \item Agile Scrum Experience
  \end{enumerate}
\end{enumerate}
\subsection{Roles and Responsibiltes}
\begin{enumerate}
\item Nathan
  \begin{enumerate}
  \item Repo Manager
  \item developers
  \end{enumerate}
\item JP
  \begin{enumerate}
    \item Scrummaster
    \item developers
  \end{enumerate}
\end{enumerate}

\section{Refferences}
\begin{enumerate}
\item Document includes references to any frameworks, simulation environments, data sets, problem descriptions,
  etc. used in this write-up
\item Citations and inline references are in IEEE format
\end{enumerate}
\printbibliography
\end{document}