\documentclass[11pt]{article}

\usepackage[backend=biber,style=ieee]{biblatex}
\usepackage{hyperref}
\hypersetup{
    colorlinks=true,
}
\addbibresource{resources.bib}

\title{Some Title}
\author{Nathan, JP}
\date{\today}

\begin{document}
\maketitle
\section{Problem Statement}
200 words max
\begin{enumerate}
\item Clearly describes the problem being addressed
\item References any external resources that support this problem (e.g. data sets, simulation/game environment,
  etc.)
\end{enumerate}

The goal of the project is to train an AI to be very good at poker(Texas Holdem), in order to do this a poker
game will be created. This game will allow human vs AI(for testing) and AI vs AI(for training).

\section{Proposed Approach}
300 words max
\begin{enumerate}
\item Identifies the type of AI/ML approach to be used to address the problem, e.g.
  \begin{enumerate}
  \item Search
  \item Classification
  \item Prediction/regression
  \item Reinforcement Learning
  \end{enumerate}
\item Describe how your problem relates to the proposed approach (i.e. how do you map the problem to the
  approach?)
\item Identify any planned libraries, resources, etc. you intend to utilize to facilitate your approach
\end{enumerate}
In order to train the model, the system will be forced to play against itself in order to improve its skills,
this unsupervised approach is used as the system does not have labeled correct plays, due to hidden information.
One possible method of training could be NEAT\cite{NEAT} where neurons are added in between generations.

\section{Team Structure}
\begin{enumerate}
\item Identifies team members
\item Identifies any relevant background of team member(s) to the problem, if applicable
\item Identifies roles and responsibilities of team members
\end{enumerate}
uhhhhhh we should probably meet about this

\section{Refferences}
\begin{enumerate}
\item Document includes references to any frameworks, simulation environments, data sets, problem descriptions,
  etc. used in this write-up
\item Citations and inline references are in IEEE format
\end{enumerate}
\printbibliography
\end{document}