\documentclass[11pt,bibliography=numbered]{article}
% This file just has the formatting rules that are used everywhere else and also sets up the title page
\usepackage{hyperref}
% \usepackage[section]{placeins}
\usepackage{tocloft}
\usepackage{geometry}
\usepackage[mmddyyyy]{datetime} %for date formatting
\usepackage{blindtext}
\usepackage[table, svgnames, dvipsnames]{xcolor}
\usepackage{makecell, cellspace, caption}
\usepackage{tabularx} % full width table
\usepackage{tikz}
\usepackage{letltxmacro}
\usepackage[utf8]{inputenc}
\usepackage{changepage}
\usepackage{titlesec}
\usepackage{enumitem}
\usepackage{float}
\usepackage[export]{adjustbox} % also loads graphicx

\usepackage[backend=biber,style=ieee]{biblatex}
\addbibresource{resources.bib}
\usepackage{geometry}

\usepackage{docmute} %has to be after biblatex
\usepackage{import}
\usepackage{datetime}
\geometry{letterpaper, margin=0.5in}

% formatting commands
% Creates a formatted image with a figure tag and caption
% 1: image filepath
% 2: Caption
% 3: figure label
\newcommand\image[3]{
  \begin{figure}[H]
    \begin{center}
      \makebox[\textwidth]{\includegraphics[max width=\textwidth]{#1}}
    \end{center}
    \caption{#2}
    \label{#3}
  \end{figure}
}
\hypersetup{
  colorlinks=true,
  linkcolor=black,
  % filecolor=magenta,      
  urlcolor=blue,
  citecolor=black
  % pdftitle={Overleaf Example},
  % pdfpagemode=FullScreen,
}
\newcommand\defaultparindent{0cm}
\newcommand\defaultparskip{0cm}
\newcommand\tablePreHeading{
  \hline
  \rowcolor{LightGray}
}
\newcommand\tablePostHeadingEnd{
  \\
  \hline
  
}
\newcommand\tableNormalEnd{
  \\
}
\newcommand\tableLastEnd{
  \\
  \hline
}
\newenvironment{noParagraphSpace}{
  \setlength{\parskip}{0cm}
}{
  \setlength{\parskip}{\defaultparskip}
}
\newenvironment{indented}{
  \begin{adjustwidth}{1cm}{}
  }{
  \end{adjustwidth}
}

\titlespacing*{\section}{0pt}{1.1\baselineskip}{\baselineskip}

% give every section a label of the same name
\begin{document}
\subimport{./}{configuration}
\pagenumbering{roman} 
\newdateformat{monthyeardate}{%
  \monthname[\THEMONTH], \THEYEAR}
\makeatletter
\def\@maketitle{%
  \newpage
  \null
  \vskip 2em%
  \begin{center}%
  \let \footnote \thanks
    {\LARGE \bf \docType \par}%
    \vskip 1.5em%
    {\LARGE \bf \@title \par}%
    \vskip 1.5em%
    {\large
      \lineskip .5em%
      \begin{tabular}[t]{c}%
        {\Large \@author}
      \end{tabular}\par}%
    \vskip 1em%
    {\Large Version: \versionNum}%
    \vskip 1em%
    {\Large \monthyeardate\@date}%
    % {\large \@date}%
    \newpage
  \end{center}%
  \par
  \vskip 1.5em}
\makeatother

\newcolumntype{1}{|l}
\newcolumntype{2}{|l}
\newcolumntype{3}{|l}
\newcolumntype{4}{|l|}
\maketitle
\begin{center}
  {\Large \bf Revision History}
  \vskip 1.5em
  \RevisionHistory
\end{center}

\renewcommand{\cftsecleader}{\cftdotfill{\cftdotsep}}
\newpage
\tableofcontents

\newpage
\pagenumbering{arabic}

\newcommand{\TODO}[1]{
  {\Huge \color{red} TODO: #1}
}

% Document Defaults: 
\newcolumntype{1}{l}
\newcolumntype{2}{l}
\newcolumntype{3}{l}
\newcolumntype{4}{l}
\setlength{\parindent}{\defaultparindent}
\setlength{\parskip}{\defaultparskip}

\LetLtxMacro{\oldsection}{\section}

\makeatletter

\renewcommand{\section}[1]{
  \oldsection{#1}
  \label{sec:#1}%
}
\subimport{./}{content}
\end{document}